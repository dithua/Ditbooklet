\documentclass[legalpaper,10pt,a5paper]{article}
%\documentclass[a4paper,10pt]{scrartcl}

\usepackage{gfsneohellenic}
\usepackage{xltxtra}
\usepackage{xgreek} % Greek hyphenation
\usepackage[scale=0.8]{geometry}
\usepackage{array}
\usepackage{amsmath}
\usepackage{pdfpages}
\usepackage{multirow}
\usepackage[usenames,dvipsnames,svgnames,table]{xcolor}
\usepackage{graphicx}
\usepackage{wallpaper}
\usepackage{hyperref}
\usepackage{bbding}
\CenterWallPaper{1}{background3.png}
\setromanfont[Mapping=tex-text]{GFS Neohellenic}
\setsansfont[Mapping=tex-text]{GFS Neohellenic}
\setmonofont[Mapping=tex-text]{GFS Neohellenic}

\newcounter{nextyear}
\renewcommand{\thesection}{\Roman{section}}
\makeatletter
\renewcommand\part{%
  \if@openright
    \cleardoublepage
  \else
    \clearpage
  \fi
  \thispagestyle{empty}%   % Original »plain« replaced by »emptyx
  \if@twocolumn
    \onecolumn
    \@tempswatrue
  \else
    \@tempswafalse
  \fi
  \null\vfil
  \secdef\@part\@spart}
\makeatother



\begin{document}
\color{black!85}
\setcounter{nextyear}{\year}
\addtocounter{nextyear}{1}
\Roman{part}
\Roman{section}
\includepdf{cover}
\tableofcontents
\newpage
\section{int main()}
Γειά σου μικρέ-ή πληροφορικάριε-α, καλωσόρισες στην κοινότητα μας! Ας γνωριστούμε λοιπόν όσο είναι δυνατόν μέσα από έναν μονόλογο και ένα κομμάτι χαρτί. Είμαστε ο φοιτητικός σύλλογος 
Πληροφορικής και Τηλεματικής. Φαντάσου πως σου μιλάει ένας τυχαίος φοιτητής του τμήματός μας που έχει την συνισταμένη των απόψεων όλων των φοιτητών. Αυτός είναι ο ρόλος μας.
Εάν μέχρι στιγμής σου φαίνονται όλα παράξενα, μην ανησυχείς, δεν θα αργήσεις να μπείς στο κλίμα! Αν και φέτος ο αριθμός τον εισακτέων ανέβηκε στους 80 από τους 45 που είχαμε, ευελπιστούμε
και είναι στο χέρι μας να διατηρήσουμε όσο γίνεται οικογενειακό το κλίμα. 
Το τμήμα μας αν και περιέχει αρκετά αντικείμενα μπορεί κανείς να πει πως έχει μια κλίση προς το διαδίκτυο (Web) και τις διαδικτυακές εφαρμογές της πληροφορικής.


\section{\#include <history.h>}
Το όνομα του Πανεπιστήμιου οφείλεται στον Παναγή Χαροκόπο ο οποίος οραματίστηκε ένα εκπαιδευτικό ίδρυμα μεάριστες κτηριακές υποδομές και εξοπλισμό, σε αρμονία με το φυσικό περιβάλλον,
που θα πρόσφερε ό,τι πιο σύγχρονο παρείχε η επιστήμη. Έτσι το 1990 ιδρύθηκε με το άρθρ. 9 του Ν.1894/90 όπως τροποποιήθηκε με το άρθρ. 17 του Ν. 1966/91 με πρωτοβουλία του 
Καθηγητή Γεωργίου Καραμπατζού (†2011). Βρίσκεσαι σε ένα τμήμα που ιδρύθηκε το 2007 και έχει βγάλει 3 φουρνιές αποφοίτων. 

Πλέον ανήκεις σε 2 κοινότητες φοιτητών! Αυτή ολόκληρου του Χαροκοπείου Πανεπιστημίου καθώς και αυτή του τμήματος μας. Κάθε κοινότητα έχει το δικό τις διοικήτικο όργανο - την συνέλευση της.
Φοιτητικό διοικητικό όργανο ολόκληρου του Πανεπιστημίου είναι η Γενική Συνέλευση, ενώ φοιτητικό διοικήτικο όργανο του τμήματος μας είναι η τμηματική συνέλευση. Περισσότερες πληροφορίες παρακάτω.

\section{Αλλαγή Περιβάλλοντος}

\subsection{Συμβολισμοί}
Στο Πανεπιστήμιο και γενικότερα στον κόσμο εκτός Λυκείου συνήθως χρησιμοποιείται άλλος συμβολισμός για πολλά από τα οποία ήδη ξέρετε. Χαρακτηριστικά παραδείγματα:
\begin{itemize}
 \item ημχ = sinx
 \item συνχ = cosx
 \item εφχ = tanx
 \item f'(x) = $\frac{df}{dx}$
\end{itemize}

\section{Εργαλεία}
\begin{itemize}

 \item Βασικός Επεξεργαστής κειμένου
 \begin{itemize}
 \item[\Checkmark] emacs
 \item[\Checkmark] vim
 \item[\Checkmark] nano
 \end{itemize}
 \item IDE's 
  \begin{itemize}
 \item[\Checkmark] Netbeans
 \item[\Checkmark] Codeblocks
 \item[\Checkmark] Eclipse
 
 \end{itemize}
 \item Git 
 \item Octave
 \item Gimp ή Photoshop
 \item Latex - Xetex
 \item Office

\end{itemize}

\section{Χρήσιμα Link}
\begin{itemize}
 \item[\Checkmark] \href{http://www.dit.hua.gr}{www.dit.hua.gr}
 \item[\Checkmark] \href{http://www.eudoxus.gr}{www.eudoxus.gr}
 \item[\Checkmark] \href{http://www.wolframalpha.com}{www.wolframalpha.com}
 \item[\Checkmark] \href{http://www.coursera.org}{www.coursera.org}
 \item[\Checkmark] \href{http://www.edx.org}{www.edx.org}
 \item[\Checkmark] \href{http://www.okeanos.gr‎}{okeanos.grnet.gr}
 \item[\Checkmark] \href{http://www.github.com}{www.github.com}
\end{itemize}


\section{Tips}



\end{document}
